\documentclass[11pt,a4paper,notitlepage]{scrreprt}
\usepackage[utf8]{inputenc}
\usepackage[ngerman]{babel}
\usepackage{ulsy}
\usepackage{tikz}
\usepackage[left=2cm,right=2cm,top=2cm,bottom=2cm]{geometry}
\usepackage[T1]{fontenc}
\usepackage{lmodern}
\usepackage{amsmath,amssymb,amstext,amsfonts,mathrsfs, amsthm}
\usepackage{graphicx}
\usepackage{geometry}
\usepackage{color}
\usepackage{enumitem}
\usepackage{framed}


% Mathmatische Symbole
\newcommand{\RR}{\mathbb{R}}
\newcommand{\EE}{\mathbb{E}}
\newcommand{\ZZ}{\mathbb{Z}}
\newcommand{\NN}{\mathbb{N}}
\newcommand{\FF}{\mathcal{F}}
\newcommand{\supp}{\operatorname{supp}}
\newcommand{\ee}{\operatorname{e}}
\newcommand{\dist}{\operatorname{dist}}
\newcommand{\grad}{\operatorname{grad}}

\newtheorem{defi}{Definition}[section]
\newtheorem{prop}[defi]{Proposition}
\newtheorem{theorem}[defi]{Theorem}
\newtheorem{cor}[defi]{Corollar}
\newtheorem{lem}[defi]{Lemma}
\newtheorem{bem}[defi]{Bemerkung}

\author{ich}
\title{Handout}



\begin{document}
\parindent 0pt


%\maketitle
%\newpage
%\tableofcontents
%\newpage

\pagestyle{empty}



 Prof. Dr. Matthes \hfill Manuela Lambacher\\
 Optimaler Transport - Geometrie und Anwendungen \hfill 09.12.2015
 \begin{center}
  {\huge{Zeit-Diskretisierung von Gradientenflüssen}} 
 \end{center}
 \vspace{4pt}
 \hrule
 
\numberwithin{equation}{section}
\renewcommand{\thechapter}{\arabic{section}}
\renewcommand{\thesection}{\arabic{section}}
\section{Approximierte Gradientenflüsse}

Betrachte \begin{eqnarray}
\dfrac{dX}{dt}=~-\grad E(X),~~~~~~~~~X(0)=X^0 \label{eq1}
\end{eqnarray} für $X(t)\in\RR^k$ für festes $t\in[0,\infty)$ und ein Energie-Funktional $E:~\RR^k\to\RR$. 
\vspace{4pt}
\begin{defi}\label{approx}
Der \textbf{approximierte Gradientenfluss} $(X_\tau)$ mit Anfangswert $X^0$ für das Energie-Funktional E in einem metrischen Raum wird folgendermaßen konstruiert:\newline
Sei $\tau > 0$ die zeitliche Schrittweite, $n\in\NN_0$, dann ist die Folge $\left( X^n_\tau \right)_{n\geq 0}$ gegeben durch:
\\
$X_\tau^0:=X^0,$
\\
$X_\tau^{n+1}$ ist ein Minimierer des Minimerungsproblems
\begin{eqnarray}
\min_X\left[E(X)+\dfrac{\dist(X_\tau^n,X)^2}{2\tau}\right] \label{Min}
\end{eqnarray}
Dann sei $X_\tau$ auf $\RR_+^k$ als stückweise konstante Funktion mit Wert $X_\tau(t)=X^n_\tau$ für $t\in [n\tau,(n+1)\tau)$ definiert. 
\end{defi}
\vspace{12pt}
Sei $\Vert \cdot \Vert$ die euklidische Norm, $E(X)\geq C$, $E(X^0)\leq\infty$ und $(X_\tau^n)_{n\in\NN_0}$ wie in Definition (\ref{approx}). Dann sind folgende Aussagen erfüllt:

\begin{lem}[Energy estimate]
\begin{eqnarray*}
\sup_{n\geq 0}E(X_\tau^n)\leq E(X^0) \label{enest}
\end{eqnarray*}
\end{lem}


\begin{lem}[Total square distance estimate]
\begin{eqnarray*}
\sum_{n\geq 0}\Vert X_\tau^n-X_\tau^{n+1}\Vert^2\leq 2\tau (E(X^0)-\inf_X E(X))\label{totalsquare}
\end{eqnarray*}
\end{lem}


\begin{lem}[Hölder 1/2-Abschätzung für $X_\tau$]  $~~$ \\
Für $s<t$ gilt:
\begin{eqnarray*}
\Vert X_\tau(s)-X_\tau(t)\Vert^2 \leq \left[\dfrac{t-s}{\tau}+1\right] \sum_{\frac{s}{\tau}\leq n \leq \frac{t}{\tau}} \Vert X^n_\tau- X_\tau^{n+1}\Vert^2 \leq 2(E(X^0)-\inf_X E(X))[(t-s)+\tau]. \label{Hölder}
\end{eqnarray*}
\end{lem}

\begin{cor}
Sei $E\in \mathcal{C}^1$. $X_\tau^m,X_\tau^{m+1}\subset(X_\tau^n)_{n\geq 0}$ erfüllen für $\omega\in\RR^k$: \\
\begin{equation}
-\grad E(X_\tau^{m+1})\cdot\omega= \dfrac{ X_\tau^m-X_\tau^{m+1}}{\tau}\cdot\omega
\end{equation}
\end{cor}

 

\newpage
\section{Der approximierte Gradientenfluss der linearen Fokker-Planck-Gleichung}

\begin{defi}
Die \textbf{lineare Fokker-Planck-Gleichung}  mit Potential \\${V(x)\in \mathcal{C}^2(\RR^k), \RR^k\to[0,\infty)}$, und Anfangswert $\rho_0\in P_{ac}(\RR^k)$ ist gegeben durch:
\begin{eqnarray}
\begin{split}
\dfrac{\partial\rho(x,t)}{\partial t}=\Delta\rho(x,t)+\nabla\cdot(\rho(x,t)\nabla V(x)),\\
\rho(x,0)=\rho_0(x).\label{FP}
\end{split}
\end{eqnarray}
Sei $T>0$. $\rho(\cdot,t)\in P_{ac}(\RR^k)$ erfüllt die (LFP) im schwachen Sinne, falls für alle $\psi\in \mathcal{C}_c^\infty(\RR^k)$ gilt:
\begin{equation*}
\int_{\RR^k} \int_0^T (\rho\Delta\psi+\rho\nabla V\cdot\nabla\psi -\rho\psi_t) dt dx=\int\rho_0\psi(x,0)dx \label{FPweak}
\end{equation*}
\end{defi}
\vspace{10pt}

\begin{defi}
Für eine Wahrscheinlichkeitsdichte $\rho(x)\in P_{ac}(\RR^k)$ und ein Potential $V(x)\in L^1(\RR^k)$ sei das \textbf{Freie Energie Funktional $E(\rho)$} gegeben durch:
\begin{equation}
E(\rho):=\int_{\RR^k} \rho\log\rho~ dx + \int_{\RR^k}\rho V dx \label{FEFktn}
\end{equation}
\end{defi}
\vspace{10pt}


\begin{theorem}
Sei $V(x)\in \mathcal{C}^2(\RR^k)$ ein Potential, sodass $V(x)=O(\vert x \vert^2)$ für $x\to\infty$. \\
Sei $\rho_0\in P_{ac}(\RR^k)$ derart, dass für das Freie Energie Funktional gilt $E(\rho_0)<\infty$. Sei $\tau>0$.\\\\
Sei außerdem $\rho_\tau(x,t)$ der (Def. \ref{approx}) approximierte Gradientenfluss für $E(\rho)$ bzgl. der quadratischen Wassersteinmetrik. 
Dann gilt: \\
\begin{equation*}
\rho_\tau(x,t)\rightharpoonup \rho(x,t) \text{ in } L^1(\RR^k) \text{ für } \tau\to 0,
\end{equation*}
wobei $\rho(x,t)\in C^\infty(\RR^k\times(0,\infty))$ die eindeutige Lösung der linearen Fokker-Planck-Gleichung mit Anfangswert $\rho_0$ ist. 
\end{theorem}

\end{document}
