\documentclass[11pt,a4paper,notitlepage]{scrreprt}
\usepackage[utf8]{inputenc}
\usepackage[ngerman]{babel}
\usepackage{ulsy}
\usepackage{tikz}
\usepackage[left=2cm,right=2cm,top=2cm,bottom=2cm]{geometry}
\usepackage[T1]{fontenc}
\usepackage{lmodern}
\usepackage{amsmath,amssymb,amstext,amsfonts,mathrsfs, amsthm}
\usepackage{graphicx}
\usepackage{geometry}
\usepackage{color}
\usepackage{enumitem}
\usepackage{framed}


% Mathmatische Symbole
\newcommand{\RR}{\mathbb{R}}
\newcommand{\EE}{\mathbb{E}}
\newcommand{\ZZ}{\mathbb{Z}}
\newcommand{\NN}{\mathbb{N}}
\newcommand{\FF}{\mathcal{F}}
\newcommand{\supp}{\operatorname{supp}}
\newcommand{\ee}{\operatorname{e}}
\newcommand{\dist}{\operatorname{dist}}
\newcommand{\grad}{\operatorname{grad}}

\newtheorem{defi}{Definition}[section]
\newtheorem{prop}[defi]{Proposition}
\newtheorem{theorem}[defi]{Theorem}
\newtheorem{cor}[defi]{Corollar}
\newtheorem{lem}[defi]{Lemma}
\newtheorem{bem}[defi]{Bemerkung}

\author{ich}
\title{Handout}



\begin{document}
\parindent 0pt


%\maketitle
%\newpage
%\tableofcontents
%\newpage

\pagestyle{empty}



 Prof. Dr. Daniel Matthes \hfill Manuela Lambacher\\
 Optimaler Transport - Geometrie und Anwendungen \hfill 09.12.2015
 \begin{center}
  {\huge{Zeit-Diskretisierung von Gradientenflüssen}} 
 \end{center}
 \vspace{4pt}
 \hrule
 
\numberwithin{equation}{section}
\renewcommand{\thechapter}{\arabic{section}}
\renewcommand{\thesection}{\arabic{section}}
\section{Approximierte Gradientenflüsse im euklidischen $\RR^k$}

Betrachte \begin{eqnarray}
\dfrac{dX}{dt}=~-\grad E(X),~~~~~~~~~X(0)=X^0, \label{eq1}
\end{eqnarray} wobei das Energie-Funktional $E:~\RR^k\to\RR$ glatt und nach unten beschränkt sei sowie ${E(X^0)<\infty}$. 
\vspace{14pt}

Variationelle Konstruktion des \textbf{approximierten Gradientenflusses} $(\bar{X}_\tau)$ mit der euklidischen Metrik:\newline
Sei $\tau > 0$ die zeitliche Schrittweite, $n\in\NN_0$, dann ist die Folge $\left( X^n_\tau \right)_{n\geq 0}$ definiert durch:
\begin{itemize}
\item[i)]$X_\tau^0:=X^0,$
\item[ii)]$X_\tau^{n+1}$ ist eine Lösung des Minimierungsproblems
\begin{eqnarray}
\min_{X\in\RR^k}\left[E(X)+\dfrac{\Vert X_\tau^n-X\Vert^2}{2\tau}\right] \label{Min}
\end{eqnarray}
\end{itemize}
Dann sei $\bar{X}_\tau:\RR_{\geq0}\to\RR^k$ eine stückweise konstante Funktion mit Wert $\bar{X}_\tau(t)=X^n_\tau$ für ${t\in [n\tau,(n+1)\tau)}$. 

\vspace{12pt}

\begin{lem}
Alle $X_\tau^m,X_\tau^{m+1}\subset(X_\tau^n)_{n\geq 0}$ erfüllen für $\omega\in\RR^k$: \\
\begin{equation}
-\grad E(X_\tau^{m+1})\cdot\omega= \dfrac{ X_\tau^{m+1}-X_\tau^m}{\tau}\cdot\omega
\end{equation}
\end{lem}

 

\begin{lem}[Energy estimate]
\begin{eqnarray*}
\sup_{n\geq 0}E(X_\tau^n)\leq E(X^0) \label{enest}
\end{eqnarray*}
\end{lem}


\begin{lem}
\begin{eqnarray*}
\dfrac{\tau}{2}\sum_{n\geq 0} \left(\dfrac{\Vert X_\tau^n-X_\tau^{n+1}\Vert}{\tau}\right)^2\leq E(X^0)-\inf_{X\in\RR^k} E(X)\label{totalsquare}
\end{eqnarray*}
\end{lem}


\begin{lem}[Hölder 1/2-Abschätzung für $\bar{X}_\tau$]  $~~$ \\
Für $s<t$ gilt:
\begin{eqnarray*}
\Vert \bar{X}_\tau(s)-\bar{X}_\tau(t)\Vert^2 \leq \left[\dfrac{t-s}{\tau}+1\right] \sum_{\frac{s}{\tau}-1\leq n \leq \frac{t}{\tau}-1} \Vert X^n_\tau- X_\tau^{n+1}\Vert^2 \leq 2(E(X^0)-\inf_{X\in\RR^k} E(X))[(t-s)+\tau]. \label{Hölder}
\end{eqnarray*}
\end{lem}

\begin{cor}
\begin{eqnarray*}
\bar{X}_{\tau_i}(t) \overset{\tau_i\to0}{\longrightarrow} X \in \mathcal{C}^{\dfrac{1}{2}}([0,\infty);\RR^k) \text{ gleichmäßig bzgl. }t\geq 0,
\end{eqnarray*}
und es gilt für alle $0\leq t_1\leq t_2<\infty$:
\begin{equation}
-\int_{t_1}^{t_2}\grad E(X(t))\cdot \omega dt = [X(t_2)-X(t_1)]\cdot\omega
\end{equation}
\end{cor}

\newpage
\section{Der approximierte Gradientenfluss der linearen Fokker-Planck-Gleichung}

Die \textbf{lineare Fokker-Planck-Gleichung}  mit Potential ${V\in \mathcal{C}^2(\RR^d)}$ und Anfangswert $\rho_0\in P_{ac,2}(\RR^d)$ ist gegeben durch:
\begin{align}\tag{FP}
\begin{split}
\dfrac{\partial\rho(x,t)}{\partial t}=\Delta\rho(x,t)+\nabla\cdot(\rho(x,t)\nabla V(x)),\\
\rho(x,0)=\rho_0(x).\label{FP}
\end{split}
\end{align}
Sei $T>0$. $\rho(\cdot,t)\in P_{ac,2}(\RR^d)$ erfüllt die (LFP) im schwachen Sinne, falls für alle $\psi\in \mathcal{C}_c^\infty(\RR^d)$ gilt:
\begin{equation}
\int_{\RR^d} \int_0^T (\rho\Delta\psi+\rho\nabla V\cdot\nabla\psi -\rho\psi_t) dt dx=\int\rho_0\psi(x,0)dx \label{FPweak}
\end{equation}

\vspace{10pt}

\begin{defi}
Für eine Wahrscheinlichkeitsdichte $\rho(x)\in P_{ac,2}(\RR^d)$ und ein Potential $V\in L^1(\RR^d)$ sei das \textbf{Freie Energie Funktional $H(\rho)$} gegeben durch:
\begin{equation}
H(\rho):=\int_{\RR^d} \rho\log\rho~ dx + \int_{\RR^d}\rho V dx \label{FEFktn}
\end{equation}
\end{defi}
\vspace{10pt}
Interpretation: (FP) entspricht $\dfrac{\partial\rho}{\partial t} = -\grad_{W_2} H(\rho)$

\begin{theorem}
Sei $V\in \mathcal{C}^2(\RR^d)$ ein Potential, sodass $V(x)=O(\vert x \vert^2)$ für $x\to\infty$. \\
Sei $\rho_0\in P_{ac}(\RR^d)$ derart, dass für das Freie Energie Funktional gilt $H(\rho_0)<\infty$. Sei $\tau>0$.\\\\
Sei außerdem $\bar{\rho}_\tau(x,t)$ der approximierte Gradientenfluss für $H(\rho)$ bzgl. der quadratischen Wassersteinmetrik. 
Dann gilt: \\
\begin{equation*}
\bar{\rho}_\tau(x,t)\rightharpoonup \rho(x,t) \text{ in } L^1(\RR^d) \text{ für } \tau\to 0,
\end{equation*}
wobei $\rho(x,t)\in C^\infty(\RR^d\times(0,\infty))$ die eindeutige Lösung der linearen Fokker-Planck-Gleichung mit Anfangswert $\rho_0$ ist. \\\\
Der approximierte Gradientenfluss $\bar{\rho}_\tau:(0,\infty)\times\RR^d\to[0,\infty)$ wird hierbei analog zu Abschnitt 1 konstruiert:
\begin{equation*}
\bar{\rho}_\tau(t):=\rho_\tau^n ~\text{ für }t\in[n\tau,(n+1)\tau)
\end{equation*}
für die Folge $(\rho_\tau^n)_n,~n\in\NN_0$:
\begin{itemize}
\item[i)]$\begin{aligned}[t]\rho_\tau^0=\rho^0; \end{aligned}$
\item[ii)]$\begin{aligned}[t]
\rho_\tau^{n+1}=\underset{\rho\in\mathcal{P}_{2}(\RR^d)}{\operatorname{argmin}}\left\{H(\rho)+\dfrac{W_2(\rho_\tau^n,\rho)^2}{2\tau}\right\}
\end{aligned}$\end{itemize}
\end{theorem}

\end{document}
