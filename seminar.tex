\documentclass[11pt,a4paper,notitlepage]{scrreprt}
\usepackage[utf8]{inputenc}
\usepackage[ngerman]{babel}
\usepackage{ulsy}
\usepackage{tikz}
\usepackage[left=2cm,right=2cm,top=2cm,bottom=2cm]{geometry}
\usepackage[T1]{fontenc}
\usepackage{lmodern}
\usepackage{amsmath,amssymb,amstext,amsfonts,mathrsfs, amsthm}
\usepackage{graphicx}
\usepackage{geometry}
\usepackage{color}
\usepackage{enumitem}
\usepackage{framed}


% Mathmatische Symbole
\newcommand{\RR}{\mathbb{R}}
\newcommand{\EE}{\mathbb{E}}
\newcommand{\ZZ}{\mathbb{Z}}
\newcommand{\NN}{\mathbb{N}}
\newcommand{\FF}{\mathcal{F}}
\newcommand{\supp}{\operatorname{supp}}
\newcommand{\ee}{\operatorname{e}}
\newcommand{\dist}{\operatorname{dist}}
\newcommand{\grad}{\operatorname{grad}}

\newtheorem{defi}{Definition}[section]
\newtheorem{prop}[defi]{Proposition}
\newtheorem{theorem}[defi]{Theorem}
\newtheorem{cor}[defi]{Corollar}
\newtheorem{lem}[defi]{Lemma}
\newtheorem{bem}[defi]{Bemerkung}

\author{ich}
\title{Seminar 0.0}



\begin{document}
\parindent 0pt


%\maketitle
%\newpage
%\tableofcontents
%\newpage

\pagestyle{plain}



 Prof. Dr. Matthes \hfill Manuela Lambacher\\
 Optimaler Transport - Geometrie und Anwendungen \hfill 09.12.2015
 \begin{center}
  {\huge{Zeit-Diskretisierung von Gradientenflüssen}} 
 \end{center}
 
 \hrule
 
\numberwithin{equation}{section}
\renewcommand{\thechapter}{\arabic{section}}
\renewcommand{\thesection}{\arabic{section}}
\section{Generalisierte Gradientenflüsse}

Wir betrachten folgenden Gradientenfluss:
\begin{eqnarray}
\dfrac{dX}{dt}=~-\grad E(X),~~~~X(0)=X^0 \label{eq1}
\end{eqnarray}
$E:\RR^k\to\RR$ sei ein glattes, nach unten beschränktes Energie-Funktional, das außerdem $E(X^0)<\infty$ erfülle. $X(t):[0,\infty)\to\RR^k$ können wir z.B. als Position gegebener Partikel interpretieren. Wir wollen von einer eindeutigen Lösung ausgehen (z.b. via Picard-Lindelöf).\newline


Statt dieses Problem explizit zu lösen, ist unser Ziel, den Gradientenfluss als Grenzwert eines zeitlich diskretisierten Problems darzustellen, sodass jeglicher expliziter Umgang mit Subdifferentialen und dem Gradientenoperator umgangen wird.\\\\

Variationelle Konstruktion:
Der \textbf{approximierte Gradientenfluss} $(\bar{X}_\tau)$ mit der euklidischen Distanz $\Vert \cdot \Vert$ sei gegeben durch:
\newline
Sei die zeitliche Schrittweite $\tau > 0$ gegeben, $n\in\NN_0$, dann ist die Folge $\left( X^n_\tau \right)_{n\geq 0}$:
\begin{itemize}
\item[i)]$X_\tau^0:=X_0,$
\item[ii)]$X_\tau^{n+1}$ ist ein Minimierer des Minimerungsproblems
\begin{eqnarray}
\min_{X\in\RR^k}\left[E(X)+\dfrac{\Vert X_\tau^n-X\Vert^2}{2\tau}\right] \label{Min}
\end{eqnarray}
\end{itemize}
Sei $\bar{X}_\tau:\RR_{\geq0}\to\RR^k$  als stückweise konstante Funktion mit Wert $X_\tau(t)=X^n_\tau$ für $t\in [n\tau,(n+1)\tau)$ definiert.


\begin{bem}
\begin{enumerate}
\item Wir setzen zunächst keine Eindeutigkeit des Minimums voraus;
\item Existenz des Minimierers muss durch geeignete Eigenschaften von $E(X)$ gesichert werden, z.B. strikte Konvexität, untere Schranke.
\end{enumerate}
\end{bem}
\vspace{3pt}

Wir beginnen damit, den Zusammenhang zwischen dem approximierten Gradientenfluss und dem Ausgangsproblem zu untersuchen.

\begin{lem}
$X_\tau^m,X_\tau^{m+1}\subset(X_\tau^n)_{n\geq 0}$ erfüllen für $\omega\in\RR^k$: \\
\begin{equation}
-\grad E(X_\tau^{m+1})\cdot\omega= \dfrac{ X_\tau^{m+1}-X_\tau^m}{\tau}\cdot\omega \label{approxgrad}
\end{equation}
\end{lem}

\begin{proof}
Dabei wird eine Strategie eingeführt, der auch - in einem etwas umfangreicherem Setting - der nächste Abschnitt über die Fokker-Planck-Gleichung folgt. \\
Wir befinden uns im $\RR^k$. Sei $\omega$ ein beliebiger (Tangenten-)Vektor auf $X_\tau^{n+1}$. Wir betrachten Perturbationen von $X_\tau^{n+1}$ "'entlang $\omega$"': Definiere für ein kleines $\varepsilon>0$ einen "'Pfad"' $\tilde{X}_\varepsilon$ folgendermaßen: \\
\begin{eqnarray}
\tilde{X}_0=X_\tau^{n+1},~~\dfrac{d\tilde{X}_\varepsilon}{d\varepsilon}\Bigg|_{\varepsilon=0} =\omega
\end{eqnarray}
Essentiell für die folgenden Schritte ist, dass die Ableitung an der Stelle $\varepsilon$ gleich Null ist. Der Einfachheit halber nehmen wir die lineare Form $\tilde{X}_\varepsilon=X_\tau^{n+1}+\varepsilon\omega$ an.\\
(Anmerkung: Wenn wir allgemeinere Perturbationen betrachten wollen, verfahren wir ganz analog, indem wir nach Taylor \begin{align}
\tilde{X}_\varepsilon = X_0+\dfrac{d\tilde{X}_\varepsilon}{d\varepsilon}\Bigg|_{\varepsilon=0}\cdot\varepsilon+O(\varepsilon^2),\\
E(\tilde{X}_\varepsilon)=E(X_\tau^{n+1})+\varepsilon\langle \grad E(X_\tau^{n+1}),\omega\rangle+O(\varepsilon^2) \label{b)}
\end{align}
verwenden.)\\\\



Da $X_\tau^{n+1}$ Minimierer ist, gilt offensichtlich
\begin{eqnarray}
E(X_\tau^{n+1})+\dfrac{\Vert X_\tau^n-X_\tau^{n+1}\Vert^2}{2\tau}\leq E(\tilde{X}_\varepsilon)+\dfrac{\Vert X_\tau^n-\tilde{X}_\varepsilon\Vert^2}{2\tau}~~~\forall \varepsilon>0 \label{a)}\\
\overset{\cdot\frac{1}{\varepsilon}}\Leftrightarrow \dfrac{E(X_\tau^{n+1})-E(\tilde{X}_\varepsilon)}{\varepsilon}\leq \dfrac{\Vert X_\tau^n-\tilde{X}_\varepsilon\Vert^2-\Vert X_\tau^n-X_\tau^{n+1}\Vert^2}{2\tau\varepsilon}\\
\Leftrightarrow -\dfrac{E(X_\tau^{n+1}+\varepsilon\omega)-E(X_\tau^{n+1})}{\varepsilon}\leq \dfrac{\Vert X_\tau^n-X_\tau^{n+1}-\varepsilon\omega\Vert^2-\Vert X_\tau^n-X_\tau^{n+1}\Vert^2}{2\tau\varepsilon}
\end{eqnarray}
Wir schreiben nun vor beide Seiten $\lim_{\varepsilon\to 0}$ und interpretieren diese Differenzenquotienten als Ableitungen nach $\varepsilon$. Wir erhalten (mit $\begin{aligned}\dfrac{d}{d\varepsilon}\Vert X_\tau^n-X_\tau^{n+1}-\varepsilon\omega\Vert^2=\dfrac{d}{d\varepsilon}(...)_1^2+(...)_2^2+...=2(...)_1\cdot(-\omega_1)+(...)\end{aligned}$) :
\begin{equation}
-\grad E(X_\tau^{n+1})\cdot\omega\leq \dfrac{\langle X_\tau^{n+1}-X_\tau^n,\omega\rangle}{\tau}
\end{equation}
Gleichheit wird durch die Betrachtung von $-\omega$ erreicht.
\end{proof}

Bevor wir $\tau \to 0$ gehen lassen und betrachten, ob ein Grenzwert existiert und tatsächlich unsere Ausgangsgleichung \ref{eq1} erfüllt, wollen wir zunächst einige Eigenschaften untersuchen. 

\subsection{Drei Ungleichungen}



$X_\tau^{n+1}$ ist ein Minimierer des Funktionals $X \mapsto E(X)+\dfrac{\Vert X_\tau^n-X\Vert^2}{2\tau}$. Damit ist offensichtlich das Funktional an der Stelle $X_\tau^{n+1}$ kleiner gleich dessen Wert an der Stelle $X_\tau^n$ ($\Vert X_\tau^n-X_\tau^n\Vert=0$). Damit haben wir
\begin{eqnarray}
E(X_\tau^{n+1})+\dfrac{\Vert X_\tau^n-X_\tau^{n+1}\Vert^2}{2\tau}\leq E(X_\tau^n). \label{ineq1}
\end{eqnarray} 
\begin{lem}[Energy estimate]
\begin{eqnarray}
\sup_{n\geq 0}E(X_\tau^n)\leq E(X^0) \label{enest}
\end{eqnarray}
\end{lem}
\begin{proof}
Folgt sofort aus (\ref{ineq1}), da der zweite Term nichtnegativ und somit die Folge der $(E(X_\tau^n))$ für steigendes n monoton abnehmend ist. 
\end{proof}

\begin{lem}
\begin{eqnarray}
\dfrac{\tau}{2}\sum_{n\geq 0} \left(\dfrac{\Vert X_\tau^n-X_\tau^{n+1}\Vert}{\tau}\right)^2\leq E(X^0)-\inf_{X\in\RR^k} E(X)\label{totalsquare}
\end{eqnarray}
Interpretation als kinetische Energie!
\end{lem}
\begin{proof}
Summation über (\ref{ineq1}), Teleskopsumme, Infimum da $E(X_\tau^n)$ monoton fallend. Infimum entweder über $X_\tau^n$ oder $X\in\RR^k$, möglich wegen unterer Schranke an Potential. 
\end{proof}

\begin{lem}[Hölder 1/2-Abschätzung für $\bar{X}_\tau$]  $~~$ \\
Für $s<t$ gilt:
\begin{eqnarray}
\Vert \bar{X}_\tau(s)-\bar{X}_\tau(t)\Vert^2 \overset{1)}\leq \left[\dfrac{t-s}{\tau}+1\right] \sum_{\frac{s}{\tau}-1\leq n \leq \frac{t}{\tau}-1} \Vert X^n_\tau- X_\tau^{n+1}\Vert^2 \overset{2)}\leq 2(E(X^0)-\inf_X E)[(t-s)+\tau]. \label{Hölder}
\end{eqnarray}
\end{lem}

\begin{proof}
\begin{itemize}
\item[2)] folgt wiederum aus Aufsummation von (\ref{ineq1}) und Monotonie.
\item[1)] Nach Definition: $\bar{X}_\tau(s)=X_\tau^n$ für $s\in [n\tau,(n+1)\tau)$. Damit gilt für $n$: $n=\lfloor \frac{s}{\tau}\rfloor$. Analog sei $\bar{X}_\tau(t)=X_\tau^m$, $m=\lfloor\frac{t}{\tau}\rfloor$.\\
\begin{align*}
\Vert \bar{X}_\tau(s)-\bar{X}_\tau(t)\Vert=&\Vert X_\tau^n-X_\tau^m\Vert \\\overset{\triangle -Ungl.}\leq&\Vert X_\tau^n-X_\tau^{n+1}\Vert+\Vert X_\tau^{n+1}-X_\tau^{n+2}\Vert+...+\Vert X_\tau^{m-1}-X_\tau^m\Vert\\
=&\sum_{n\leq k\leq m-1}\Vert X_\tau^k-X_\tau^{k+1}\Vert=\sum_{\frac{s}{t}-1\leq k\leq \frac{t}{\tau}-1}\Vert X_\tau^k-X_\tau^{k+1}\Vert\\
\Vert \bar{X}_\tau(s)-\bar X_\tau(t)\Vert^2\leq&\Bigg(\sum_{\frac{s}{t}-1\leq k\leq \frac{t}{\tau}-1}\Vert X_\tau^k-X_\tau^{k+1}\Vert\Bigg)^2 \\ \overset{CS}\leq&\Bigg(\sum_{\frac{s}{t}-1\leq k\leq \frac{t}{\tau}-1}\Vert X_\tau^k-X_\tau^{k+1}\Vert^2\Bigg)\underset{\frac{t}{\tau}-(\frac{s}{\tau}-1)}{\underbrace{\Bigg(\sum_{\frac{s}{t}-1\leq k\leq \frac{t}{\tau}-1}1\Bigg)}}
\end{align*}
\end{itemize}
\end{proof}




\begin{bem}
Wir haben hier nie die besonderen Eigenschaften der euklidischen Norm benutzt! Diese Ungleichungen gelten auch bei approximierten Gradientenflüssen jeder beliebigen anderen Norm.
\end{bem}

Diese drei Abschätzungen stellen uns die relative Kompaktheit von $(\bar{X}_\tau)$ für $\tau \to 0$ sicher, und damit die Existenz einer konvergenten Teilfolge.  \\


\subsection{Grenzwert erfüllt Gradientenfluss}
\begin{cor}
\begin{eqnarray*}
\bar{X}_\tau(t) \overset{\tau\to0}{\longrightarrow} X \in \mathcal{C}^{1/2}([0,\infty);\RR^k) \text{ gleichmäßig bzgl. }t\geq 0,
\end{eqnarray*}
und es gilt für alle $0\leq t_1\leq t_2<\infty$:
\begin{equation}
-\int_{t_1}^{t_2}\grad E(X(t))\cdot \omega dt = [X(t_2)-X(t_1)]\cdot\omega
\end{equation}
\end{cor}

\begin{proof}
Seien $t_1,t_2>0$. Wir beginnen mit (\ref{approxgrad}) und summieren von $n_1=\lfloor\frac{t_1}{\tau}\rfloor$ bis $n_2=\lfloor\frac{t_2}{\tau}\rfloor-1$:\\
\begin{align}
\sum_{n_1}^{n_2}\dfrac{1}{\tau} (X_\tau^{m+1}-X_\tau^m)\cdot\omega=\sum_{n_1}^{n_2}-\grad E(X_\tau^{m+1})\cdot\omega
\end{align}
Beachte für die rechte Seite, dass $\forall n\in\NN_0$
\begin{align*}
-\grad E(X^n)=\dfrac{1}{\tau}\int_{n\tau}^{n\tau+\tau}-\grad E(\bar X_\tau(t))dt.
\end{align*}
Damit haben wir - linke Seite Teleskopsumme -
\begin{align}
\dfrac{1}{\tau}(X_\tau^{n_2+1}-X_\tau^{n_1})\cdot\omega=\dfrac{1}{\tau}\int_{(n_1+1)\tau}^{(n_2+2)\tau}-\grad E(\bar{X}_\tau(t))\cdot\omega dt\\
\Leftrightarrow (\bar{X}_\tau(t_2)-\bar{X}_\tau(t_1))\cdot\omega = \int_{t_1}^{t_2}-\grad E(\bar{X}_\tau(t))\cdot\omega dt+O(\tau),
\end{align}
wobei wir den Restterm $O(\tau)$ auch explizit schreiben können:
\begin{align*}
-\int_{t_1}^{(n_1+1)\tau}-\grad E(\bar X_\tau(t))\cdot\omega dt+\int_{t_2}^{(n_2-2)\tau}-\grad E(\bar X_\tau(t))\cdot\omega dt\\
=+\grad E(\bar X_\tau(t_1))\underset{\leq\tau}{\underbrace{((n_1+1)\tau-t_1)}}-\grad E(\bar X_\tau(t_2))\underset{\leq\tau}{\underbrace{((n_2+2)\tau-t_2)}}=O(\tau)
\end{align*}
Die vorherigen Ungleichungen sichern uns die Existenz einer konvergenten (Teil-)Folge von $\bar{X}_\tau$ für $\tau\to 0$ (genäherte gleichgradige Stetigkeit; Beschränktheit; $1/2$ wegen Hölder! Art Ascola-Aszoli.). $-\grad E$ ist glatt und konvergiert damit ebenfalls gegen $-\grad E(X)$. $O(\tau)$ konvergiert natürlich gegen Null. Wir bekommen damit eine schwache Formulierung unseres ursprünglichen Gradientenflusses (\ref{eq1})!
\begin{equation*}
\int_{t_1}^{t_2}\grad E(X(t))\cdot \omega dt = [X(t_2)-X(t_1)]\cdot\omega
\end{equation*}
Und das ist genau das Ziel dieses Vortrages. Wir sehen: variationelle Konstruktion konvergiert gegen richtige Lösung.  
\end{proof}

 

\newpage
\section{Die lineare Fokker-Plank-Gleichung, Monge-Kantorovich und der approximierte Gradientenfluss}

(Notation in diesem Abschnitt: $\rho=\rho(x,t)$)\\\\
Wir können diesen Ansatz jetzt dazu verwenden, zu partiellen Differentialgleichungen passende Energiefunktionale in beliebigen metrischen Räumen zu suchen. Diese Strategie wurde zum ersten Mal am Beispiel der linearen Fokker-Plank-Gleichung angewandt. \\
\begin{defi}
Die \textbf{lineare Fokker-Planck-Gleichung} (FP) mit glattem Potential $V:\RR^d\to[0,\infty)$ und Anfangswert $\rho_0\in\mathcal{P}_{ac,2}$, absolut stetige Wahrscheinlichkeitsdichte mit zweitem endlichen Moment auf $\RR^d$, ist gegeben durch:
\begin{eqnarray}
\begin{split}
\dfrac{\partial\rho}{\partial t}=\Delta\rho+\nabla\cdot(\rho\nabla V),\\
\rho(0)=\rho_0.\label{FP}
\end{split}
\end{eqnarray}
Alle Lösungen $\rho(x,t)$ sind ebenfalls abs. stetige Wahrscheinlichkeitsdichten auf $\RR^d$ für fast alle festen $t$, d.h. $\rho(x,t)\geq 0$ für fast alle $(x,t)\in \RR^d\times(0,\infty)$  und $\int_{\RR^d}\rho(x,t)dx=1$ für fast alle $t\in(0,\infty)$
Eine Wahrscheinlichkeitsdichte $\rho\in\mathcal{P}_{ac,2}$ erfüllt die (LFP) im schwachen Sinne, falls für alle Testfunktionen $\psi$ gilt (Herleitung wie üblich, Testfunktion und aufintegrieren):
\begin{equation}
\int_{\RR^d} \int_T (\rho\Delta\psi+\rho\nabla V\cdot\nabla\psi -\rho\psi_t) dt dx=\int_{\RR^d}\rho_0\psi(x,0)dx \label{FPweak}
\end{equation}
\end{defi}

Die Fokker-Planck Gleichung beschreibt die zeitabhängige Entwicklung einer Wahrscheinlichkeitsdichte für einen stochastischen Prozess. Sie wird verwendet, um die Wahrscheinlichkeitsdichte für Ort oder Geschwindigkeit eines Partikels zu beschreiben, das verschiedenen Kräften ausgesetzt ist.

\begin{bem}
Wenn $V$ geeignete Wachstumsbedingungen erfüllt, existiert eine eindeutige stationäre Lösung zur (LFP), die sog. Gibbsche Verteilung.
\end{bem}
Auch wenn die explizite Lösung der Gleichung bereits bekannt war, liefert die Anwendung des hier zu zeigenden Approximationsverfahren interessante Einblicke in die Methode, die auch für andere Diffgleichungen und in numerischer Implementierung Anwendung findet. 


Wir konstruieren ein zeitlich diskretisiertes, iteratives variationelles Verfahren (d.h. die Lösung minimiert ein bestimmtes konvexes Funktional), dessen Lösungen gegen die Lösung der linearen Fokker-Planck-Gleichung konvergieren.

\begin{defi}
Für eine Wahrscheinlichkeitsdichte $\rho\in P_{ac}(\RR^d)$ und ein Potential $V\in\mathcal{L}^1(\RR^d)$ sei das \textbf{Freie Energie Funktional $H(\rho)$} gegeben durch:
\begin{equation}
H(\rho):=\int_{\RR^d} \rho\log\rho dx + \int_{\RR^d}\rho V dx \label{FEFktn}
\end{equation}
\end{defi}

\begin{bem}
\begin{itemize}
\item Das Freie Energie Funktional ist wohldefiniert für Lösungen $\rho(x,t)$ der (LFP), falls $H(\rho_0)<\infty$.
\item $H(\rho(x,t))$ ist für Lösungen des (LFP) bzgl. der Zeit monoton abnehmend.\end{itemize}
\item Nach der Definition des letzten Vortrages betrachten wir die Summe aus interner und potentieller Energie. Alternativ: erster Term (Shannon)-Entropie (Maß für Unsicherheit).
\end{bem}

Wir wollen zeigen, dass die Lösung der Gleichung der Richtung des steilsten Abstieges bzgl. der Wassersteinmetrik des Funktionals folgt. \\Otto:(FP) entspricht $\dfrac{\partial\rho}{\partial t} = -\grad_{W_2} H(\rho)$\\

Wiederholung:
\begin{defi}
Seien $f,g\in\mathcal{P}(\RR^d)$ zwei Wahrscheinlichkeitsmaße (Notation: bzw. ihre assoziierten Wahrscheinlichkeitsdichten.)
\begin{equation}
W_2(f,g):=\inf_{\pi \in \Pi(f,g)} \left\{ \int_{\RR^d\times\RR^d} \vert x-y \vert^2 d\pi(x,y)\right\}^\frac{1}{2}.\label{W2}
\end{equation}
$\Pi(f,g)$: Menge aller multivariaten Wahrscheinlichkeitsmaße auf $\RR^d \times \RR^d$ mit Randverteilungen f und g.
\end{defi}
Da die Wassersteinmetrik zusammen mit dem Raum aller Wahrscheinlichkeitsmaße aber keine Riemann'sche Mannigfaltigkeit ist, und damit der Gradient zunächst nicht wohldefiniert ist, müssen wir analog zum ersten Teil des Vortrags zeitlich diskretisieren.\\


\begin{theorem}
Sei $V\in \mathcal{C}^2(\RR^d)$ ein Potential, sodass $V(x)=O(\vert x \vert^2)$ für $x\to\infty$. \\
Sei $\rho_0\in P_{ac,2}(\RR^d)$ derart, dass für das Freie Energie Funktional gilt $H(\rho_0)<\infty$. Sei $\tau>0$.\\\\
Sei außerdem $\bar{\rho}_\tau(x,t)$ der approximierte Gradientenfluss für $H(\rho)$ bzgl. der quadratischen Wassersteinmetrik. 
Dann gilt: \\
\begin{equation*}
\bar{\rho}_\tau(x,t)\rightharpoonup \rho(x,t) \text{ in } L^1(\RR^d) \text{ für } \tau\to 0,
\end{equation*}
wobei $\rho\in C^\infty(\RR^d\times(0,\infty))$ die eindeutige Lösung der linearen Fokker-Planck-Gleichung mit Anfangswert $\rho_0$ ist. D.h. der approximiert Gradientenfluss konvergiert schwach gegen die Lösung der Fokker-Planck-Gleichung.\\\\
Der approximierte Gradientenfluss $\bar{\rho}_\tau:(0,\infty)\times\RR^d\to[0,\infty)$ wird hierbei analog zu Abschnitt 1 konstruiert:
\begin{equation*}
\bar{\rho}_\tau(t):=\rho_\tau^n ~\text{ für }t\in[n\tau,(n+1)\tau)
\end{equation*}
für die Folge $(\rho_\tau^n)_n,~n\in\NN_0$:
\begin{eqnarray}
\rho_\tau^0:=\rho^0;^\\
\rho_\tau^{n+1}=\underset{\rho\in\mathcal{P}_{2}(\RR^d)}{\operatorname{argmin}}\left\{H(\rho)+\dfrac{W_2(\rho_\tau^n,\rho)^2}{2\tau}\right\} \label{rho^n}
\end{eqnarray}
\end{theorem}

Das heißt eben genau: Die Fokker-Planck-Gleichung ist der Gradientenfluss für das Freie-Energie Funktional mit der Wassersteindistanz.

\subsection{Der Beweis}

\begin{proof}
\begin{enumerate}
\item Da H strikt konvex im üblichen Sinne ist, ebenso wie die Wassersteinmetrik konvex ist, ist der Minimierer in (\ref{rho^n})  eindeutig. Auch möglich: nach dem letzten Vortrag ist für $V$ konvex das Funktional displacement konvex. Warum überhaupt ein Minimierer existiert: Würde über den heutigen Vortrag hinausgehen. Distanz muss mehr wachsen als H gegen $-\infty$ gehen für flache $\rho$.
\item Analog zu (\ref{enest}) erhält man eine obere Grenze für unser Funktional:\\ ${\sup_{t\geq 0} H(\bar{\rho}_\tau(t))=\sup_n H(\rho_\tau^n)\leq H(\rho_0)}$.\\
Dies garantiert die Straffheit der Familie $(\bar{\rho}_\tau)_{\tau >0}$ in der x-Variable sowie schwache $L^1$-Kompaktheit.\\
Die Hölder 1/2 Ungleichung (\ref{Hölder}) hinsichtlich der Wassersteindistanz sichert genäherte gleichgradige Stetigkeit bzgl. $t$ von $(\bar{\rho}_\tau)_\tau$.\\ ($W_2(\bar{\rho}_\tau(s)-\bar{\rho}_\tau(t))<\varepsilon \forall \vert s-t\vert<\delta$; $\forall \varepsilon$ setze $\delta=\frac{\varepsilon^2}{2(H(\rho_0)-\inf H)}-\tau$)\\\\
$\Rightarrow$ All dies ist hinreichend, damit(betrachte falls nötig Teilfolge)$(\bar{\rho}_\tau)_\tau$ zu einer Funktion $\rho: \RR_{\geq0} \mapsto P_{ac,2}(\RR^d)$ konvergiert, stetig falls $P_{ac,2}(\RR^d)$ mit der schwachen $L^1$-Topologie verknüpft ist. \\
\item Erfüllt nun dieser Grenzwert den Satz, d.h. die (FK)? \\
Wir betrachten eine geeignete Variation von $\rho_\tau^{n+1}$. Ein wenig komplexer als in Abschnitt 1 (dort war es $X_\tau^{n+1}+\varepsilon\omega$), um unserer Wassersteindistanz gerecht zu werden, jedoch mit dem gleichen Ziel.\\
Sei $\xi\in\mathcal{C}_c^\infty(\RR^d,\RR^d)$ ein glattes Vektorfeld mit kompaktem Träger, und $T_\varepsilon$ eine Familie von Kurven verknüpft mit dem Vektorfeld $\operatorname{Id} +\varepsilon\xi$. Dann definiere folgende Perturbation:
\[\tilde{\rho}_\varepsilon=T_\varepsilon \#\rho_\tau^{n+1} \]
(Pushforward: $(\tilde{\rho}_\varepsilon \mathcal{L}^d)(B)=(\rho_\tau^{n+1}\mathcal{L}^d)(T_\varepsilon^{-1}(B))$)\\
Für $\varepsilon$ klein genug ist $T_\varepsilon$ ein $C^1$-Diffeomorphismus und invertierbar ($\det(\nabla T_\varepsilon)>0$).\\
\item Es gilt aufgrund der Invertierbarkeit von $T_\varepsilon$ mithilfe von Substitution und McCann-Theorem (Vortrag Alex):  
\begin{align}
H(\tilde{\rho}_\varepsilon)=&\int\tilde{\rho}_\varepsilon\log\tilde{\rho}_\varepsilon+\int\tilde{\rho}_\varepsilon V\\
=&\int \rho_\tau^{n+1}\log\dfrac{\rho_\tau^{n+1}}{\det(I_k\varepsilon \nabla \xi)}+\int \rho_\tau^{n+1}(x)V(x+\varepsilon\xi(x))dx.
\end{align}
\item Aus der Energy Estimate und der Definition des Minimierungsproblems folgt andererseits, dass $\rho_\tau^n, \rho_\tau^{n+1}$ absolut stetig (bzgl. Lebesgue-maß)) sind. Damit (Brenier, ebenfalls Vortrag Alex) existert eine bzgl. der Wassersteinmetrik optimale Abbildung $\nabla \phi$ sodass $\nabla \phi \# \rho_\tau^n=\rho_\tau^{n+1}$, also gilt
\begin{equation}
W_2(\rho_\tau^n,\rho_\tau^{n+1})^2=\int \rho_\tau^n(x)\vert x-\nabla\phi(x)\vert^2 dx.
\end{equation}
Dann können wir $\tilde{\rho_\varepsilon}$ auch in Abhängigkeit von $\rho_\tau^n$ statt $\rho_\tau^{n+1}$ schreiben: $\tilde{\rho}_\varepsilon=[(\operatorname{Id}+\varepsilon\xi)\circ\nabla\phi]\#\rho_\tau^n$.\\
Wir verwenden nun nach der Definition der Wassersteinmetrik analog zum letzten Schritt die bestimmte Abbildung $[(\operatorname{Id}+\varepsilon\xi)\circ\nabla\phi]$ zwischen $\tilde{\rho}_\varepsilon$ und $\rho_\tau^n$, diesmal haben wir allerdings keinen Anspruch darauf, dass dies tatsächlich die optimale wäre, und erlangen so folgende Ungleichung:
\begin{equation*}
W_2(\rho_\tau^n,\tilde{\rho}_\varepsilon)^2 \leq \int \vert x-\underset{[(\operatorname{Id}+\varepsilon\xi)\circ\nabla\phi](x)}{\underbrace{\nabla \phi(x)-\varepsilon\xi\circ\nabla\phi(x)}}\vert^2 \rho_\tau^n(x)dx
\end{equation*}
Damit kommen wir auf
\begin{align}
\begin{split}
0\leq\dfrac{W_2(\rho_\tau^n,\tilde{\rho}_\varepsilon)^2}{2\tau}&+H(\tilde{\rho}_\varepsilon)-\dfrac{W_2(\rho_\tau^n,\rho_\tau^{n+1})^2}{2\tau}-H(\rho_\tau^{n+1})\\
\overset{\log(\frac{a}{b})=\log a-\log b}\leq&\int\rho_\tau^n(x)\left(\dfrac{\vert x-\nabla \phi(x)-\varepsilon\xi\circ\nabla\phi(x)\vert^2-\vert x-\nabla\phi(x)\vert^2}{2\tau}\right)dx\\
+\int\rho_\tau^{n+1}&(x)[V(x+\varepsilon\xi(x))-V(x)]dx-\int\rho_\tau^{n+1}(x)\log( \det(I_k+\varepsilon\nabla\xi(x)))dx \label{inequ}
\end{split}
\end{align}
wobei die Nichtnegativität aus der Minimierungseigenschaft von $\rho_\tau^{n+1}$ folgt.\\
\item Nun dividieren wir durch $\varepsilon$ und lassen dieses gegen $0^+$ gehen, analog zu Abschnitt 1, Beweis des 1.Lemmas. Grenzwerte und Integrale dürfen wir vertauschen, da unsere Funktionen beschränkt bzw. Wahrscheinlichkeitsmaße endlichen zweiten Moments sind (deswegen auch die Wachstumsbedingungen!!) Die sich ergebenden "'Differenzenquotienten"' formulieren wir als Ableitungen an der Stelle $\varepsilon=0$. Damit wird (\ref{inequ}) zu: 
\begin{align*}
0\leq& \lim_{\varepsilon\to 0^+} \Bigg(\dfrac{1}{2\tau} \int\left(\dfrac{\vert x-\nabla \phi(x)-\varepsilon\xi\circ\nabla\phi(x)\vert^2-\vert x-\nabla\phi(x)\vert^2}{\varepsilon}\right)\rho_\tau^n(x)dx\\
+&\int\frac{1}{\varepsilon}[V(x+\varepsilon\xi(x))-V(x)]\rho_\tau^{n+1}(x)dx-\int\dfrac{\log( \det(I_k+\varepsilon\nabla\xi(x)))-\log(\det I_k)}{\varepsilon}\rho_\tau^{n+1}(x)dx \Bigg)\\
=& \dfrac{1}{2\tau} \int \dfrac{\partial}{\partial\varepsilon}\vert x-\nabla\phi-\varepsilon\xi\circ\nabla\phi(x)\vert^2\Big|_{\varepsilon=0} \rho_\tau^n(x)dx
+\int\frac{\partial}{\partial\varepsilon}[V(x+\varepsilon\xi(x))]\Big|_{\varepsilon=0}\rho_\tau^{n+1}(x)dx\\-&\int\dfrac{\partial}{\partial\varepsilon}[\log(\underset{=1+\operatorname{tr}\varepsilon\nabla\xi(x)+O(\varepsilon^2(\nabla\xi(x)))^2}{\underbrace{\det(I_k+\varepsilon\nabla\xi(x))}})]\Big|_{\varepsilon=0} \rho_\tau^{n+1}(x)dx \\
=&\frac{1}{\tau} \int \rho_\tau^n(x)\left\langle\nabla\phi(x)-x,\xi\circ\nabla\phi(x)\right\rangle dx+\int\rho_\tau^{n+1}(x)\langle\nabla V(x),\xi(x)\rangle dx-\int\rho_\tau^{n+1}(x)1\cdot\underset{\nabla\cdot\xi}{\underbrace{(\operatorname{tr}\nabla\xi)}}(x)dx
\end{align*}
Da dies genauso für $-\xi$ gilt, bekommen wir Gleichheit. Umformuliert:
\begin{equation}
\frac{1}{\tau} \int \rho_\tau^n(x)\left\langle\nabla\phi(x)-x,\xi\circ\nabla\phi(x)\right\rangle dx\\=\int\rho_\tau^{n+1}(x)\left[(\nabla\cdot \xi)(x)-\langle\nabla V(x),\xi(x)\rangle \right]dx \label{eqwithxi}
\end{equation}
\item Jetzt sei $\xi$ derart, dass für ein $\psi\in \mathcal{C}^\infty_c(\RR^d)$ gilt $\xi=\nabla\psi$. Es gilt (Taylor):
\begin{equation}
\psi(\nabla\phi(x))-\psi(x)=\left\langle\nabla\phi(x)-x,\nabla\psi\circ\nabla\phi(x)\right\rangle+O(\vert x-\nabla\phi(x)\vert^2
\end{equation}
und damit kann die linke Seite von (\ref{eqwithxi})ersetzt werden durch
\begin{align*}
\frac{1}{\tau}\left(\int \rho_\tau^n(x)\psi\circ\nabla\phi(x)dx-\int\rho_\tau^n(x)\psi(x)dx\right)+O\left(\frac{1}{\tau}\int\rho_\tau^n(x)\vert x-\nabla\phi(x)\vert^2dx\right)\\
\overset{\nabla\phi\#\rho_\tau^n=\rho_\tau^{n+1}}=\frac{1}{\tau}\left(\int\rho_\tau^{n+1}\psi-\int\rho_\tau^n\psi\right)+O\left(\dfrac{W_2(\rho_\tau^n,\rho_\tau^{n+1})^2}{\tau}\right)
\end{align*}
\item Nun setzt man dies in (\ref{eqwithxi}) ein und es wird aufsummiert von $n_1=\lfloor t_1/\tau\rfloor$ bis zu $n_2=\lfloor t_2/\tau\rfloor-1$ für $t_1,t_2\in[0,\infty)$ beliebige Zeiten (Grenzen: siehe Abschnitt 1, Beweis Corollar):
\begin{eqnarray}
\dfrac{1}{\tau}\Bigg(\int\bar{\rho}_\tau(t_2)\psi dx-\int\bar{\rho}_\tau(t_1)\psi dx+\underset{(*)}{\underbrace{O\left(\sum_{n=n_1}^{n_2}W_2(\rho_\tau^n,\rho_\tau^{n+1})^2\right)}}\Bigg)\\
=\sum_{n_1}^{n_2}\int\rho_\tau^{n+1}(x)\left[(\nabla\cdot \nabla\psi)(x)-\langle\nabla V(x),\nabla\psi(x)\rangle \right]dx \\=\dfrac{1}{\tau}\left(\int_{t_1}^{t_2}\int\bar{\rho}_\tau(t)(\Delta\psi-\nabla V\cdot\nabla\psi)dx dt+O(\tau)\right)
\end{eqnarray}
\begin{itemize}
\item Wobei $O(\tau)$ den gerundeten Summationsgrenzen $n_1,n_2$ geschuldet ist, wie bereits in Abschnitt 1; Der Fehlerterm sind wiederum zwei Integrale über ein mit $\tau\to0$ verschwindendes Intervall, die Funktionen im Inneren sind $C_c^\infty$ bzw. $\bar{\rho}_\tau(t)=\bar{\rho}_\tau(t,\cdot)$ ist Wahrscheinlichkeitsmaß auf $\RR^d$, sodass uns z.B. Hölder eine erforderliche Schranke liefert.
\item  
$(*)=O(2\tau(H(\rho_0)-H(\rho_\tau^{n_2+1})))=O(\tau)$ wie in kin.Energie Ungl etc.
\end{itemize}
Damit gilt schließlich für alle festen $t_1,~t_2\in [0,\infty)$:
\begin{equation}
\int\bar\rho_\tau(t_2)\psi-\int\bar\rho_\tau(t_1)\psi+O(\tau)=\int_{t_1}^{t_2}\int\bar\rho_\tau(t)(\Delta\psi-\nabla V\cdot\nabla\psi)dt+O(\tau)
\end{equation}
\item Grenzwert $\tau\to 0$ liefert:
\begin{equation}
\int\rho(t_2)\psi-\int\rho(t_1)\psi=\int_{t_1}^{t_2}\int\rho(t)(\Delta\psi-\nabla V\cdot\nabla\psi)dt
\end{equation}
Setze $t_1=0$: Der Grenzwert der Approximation erfüllt nach diesem Beweis die schwache Formulierng der linearen Fokker-Planck-Gleichung mit Potential V. Da die Lösung eindeutig ist, und bekanntermaßen eine klassische Lösung exisitert, haben wir die (FP) gelöst. 


\end{enumerate}
\end{proof}


\end{document}
